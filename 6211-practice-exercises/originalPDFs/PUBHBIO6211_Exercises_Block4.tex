%%% PUBHBIO 6211 - PRACTICE EXERCISES
\documentclass[17pt,landscape]{foils}

% Include command file
\usepackage{commandsExercises}
 
 %print numbers
%\toggletrue{solutions}
\togglefalse{solutions}

% Exercises
\begin{document}

\exercise{4}{
We wish to test the relative effects of three drugs (numbered 1, 2, 3) on the reduction of fever. We randomly assign 5 children with a fever to each of the three drugs. Reduction in fever (in degrees) is noted 4 hours after administration of drug. (Note: a positive value for the variable \texttt{reduction} means the fever went down.) Our interest is in predicting reduction in fever with drug type. Stata code is run and is shown below. The model fit is: $E(REDUCTION) = \beta_0 + \beta_1 DRUG2 + \beta_2 DRUG3$
}

\begin{stata}
. generate drug1 = (drug==1) if !missing(drug)
. generate drug2 = (drug==2) if !missing(drug)
. generate drug3 = (drug==3) if !missing(drug)

. regress reduction drug2 drug3

      Source |       SS           df       MS      Number of obs   =        15
-------------+----------------------------------   F(2, 12)        =      6.79
       Model |  5.82933309         2  2.91466655   Prob > F        =    0.0106
    Residual |  5.14800001        12  .429000001   R-squared       =    0.5310
-------------+----------------------------------   Adj R-squared   =    0.4529
       Total |  10.9773331        14  .784095222   Root MSE        =    .65498

------------------------------------------------------------------------------
   reduction |      Coef.   Std. Err.      t    P>|t|     [95% Conf. Interval]
-------------+----------------------------------------------------------------
       drug2 |      -1.16   .4142463    -2.80   0.016    -2.062565   -.2574348
       drug3 |      -1.44   .4142463    -3.48   0.005    -2.342565   -.5374348
       _cons |       1.52   .2929164     5.19   0.000       .88179     2.15821
------------------------------------------------------------------------------
\end{stata}

(a) What is the correct interpretation of the estimated intercept in this model?

\answer{2cm}{The estimated mean reduction in fever for a child taking Drug 1 is 1.52 degrees.}

(b) Is the test of the intercept meaningful in this model? If yes, write a one sentence interpretation of the test result. If not, explain why it is not meaningful.

\answer{2cm}{Yes, meaningful; tests whether there is a non-zero mean reduction in fever for children taking Drug 1.\\
There is evidence of a significant mean reduction in fever for children taking Drug 1 (p$<$0.0005).}

(c) Write a one sentence interpretation of the estimated coefficient for \texttt{drug2}.

\answer{2cm}{The estimated difference in mean fever reduction for Drug 2 is 1.16 degrees lower than the mean for Drug 1.}

(d) In terms of the estimated regression coefficients $(\hat{\beta}_0, \hat{\beta}_1, \hat{\beta}_2)$, what is the estimated mean difference in fever reduction for Drug 3 minus Drug 2? Plugging in the values of these coefficients, what is this difference?

\answer{3cm}{$\hat{\beta}_2 - \hat{\beta}_1= -1.44 - (-1.16) = -0.28$\\
(Drug 2 has larger fever reduction)}

(e) Calculate the estimated mean fever reduction for Drug 2 and for Drug 3 and use these means to confirm your answer to (c).

\answer{2cm}{$\hat{E}(REDUCTION | Drug 2) = \hat{\beta}_0 + \hat{\beta}_1 = 1.52 - 1.16 = 0.36$\\
$\hat{E}(REDUCTION | Drug 3) = \hat{\beta}_0 + \hat{\beta}_2 = 1.52 - 1.44 = 0.08$\\
Difference, Drug 3 $-$ Drug 2 = $0.08 - 0.36 = -0.28$}

%%%
\newpage

(f) Can we tell from the given Stata output whether the true mean fever reduction for Drug 2 is significantly different from the true mean fever reduction for Drug 3? If yes, are they significantly different? If not, what additional Stata commands would you have to run to get the answer?

\answer{2cm}{No, we have an estimate of the mean difference but not a test. Several options to get the test result:\\
(1) Use the \texttt{lincom} command: \texttt{lincom drug3 - drug2}\\
(2) Run a new model with Drug 2 as the reference group and use the test of the Drug 3 coefficient\\
(3) Run a new model with Drug 3 as the reference group and use the test of the Drug 2 coefficient
}

(g) Do we need a partial F-test in order to test whether drug type is a significant predictor of fever reduction? If not, what information on the Stata output can we use for this test?

\answer{2cm}{We do not need a partial F-test. Since drug type is the only predictor, we can use the Overall F-test, since this tests $H_0: \beta_1=\beta_2=0$, which is a test of whether drug is a significant predictor.}

(h) If I rerun the model, but make Drug 3 the reference group, which of the following will change and which will stay the same: Overall F-test, $R^2$, Adjusted $R^2$, Intercept estimate, Coefficient for \texttt{drug2}?

\answer{1.5cm}{Stay the same: Overall F-test, $R^2$, Adjusted $R^2$ \quad 
Change: Intercept estimate, Coefficient for \texttt{drug2}}

(i) Suppose we also have the weight of the children (lbs), and add this to our regression model, so that the model we are fitting is:\\
$E(REDUCTION) = \beta_0 + \beta_1 DRUG2 + \beta_2 DRUG3 + \beta_3 WEIGHT$\\
Interpret each of the parameters in this new model.

\answer{1.5cm}{$\beta_0$ = expected mean fever reduction for children taking Drug 1 who weigh 0 pounds.\\
$\beta_1$ = expected difference in mean fever reduction, Drug 2 minus Drug 1, controlling for weight.\\
$\beta_2$ = expected difference in mean fever reduction, Drug 3 minus Drug 1, controlling for weight.\\
$\beta_3$ = expected change in mean fever reduction for a 1 lb increase in weight, holding drug constant.}

%%%
\newpage

\exercise{4}{A survey of a random sample of students at the University of New Hampshire was conducted. We are interested in predictors of grade point average (GPA), which is measured on a 4-point scale.

One variable we are interested in is \texttt{religion}, which captures a student's religious preference (1=Protestant, 2=Catholic, 3=Other). We also have as predictors number of hours studied per week (\texttt{study}), age (years), and sex (\texttt{gender}; 1=male, 0=female).

A regression model was fit to predict GPA using these variables. Use the Stata output provided at the end of the problem (note: output spans 2 pages) to answer the questions below.}

(a) What category is the reference group for the \texttt{religion} variable?

\answer{1cm}{Group 3 -- ``Other" religion (dummy variables for groups 1 and 2 are included as predictors, leaving group 3 as the reference group)}

(b) Interpret the coefficient for the dummy variable \texttt{religion1}.

\answer{2cm}{The estimated mean GPA for Protestant students is 0.0659 points lower than the estimated mean for students of ``Other" religion, controlling for hours studied, age, and gender.}

(c) Interpret the coefficient for the dummy variable \texttt{religion2}.

\answer{2cm}{The estimated mean GPA for Catholic students is 0.174 points lower than the estimated mean for students of ``Other" religion, controlling for hours studied, age, and gender.}

(d) Interpret the coefficient for \texttt{study}. Is there a significant effect of hours studied on GPA?

\answer{2cm}{$\hat{\beta}_{study}=0.0123, \text{p-value} < 0.0005$. The estimated mean GPA increases by 0.0123 points for each 1 hour studied per week, controlling for religion, age, and gender. This effect is significant (p$<$0.0005).}

%%%
\newpage

(e) What is the estimated mean difference in GPA between Protestant and Catholic students? Be sure to indicate which group has the higher GPA.

\answer{2cm}{Catholic $-$ Protestant = $\hat{\beta}_{religion2} - \hat{\beta}_{religion1} = -0.1738 - (-0.0659) = -0.1079$\\
Catholic students have 0.1079 points lower estimated mean GPA than Protestant students.\\
Could also look at the results of the \texttt{lincom} command: difference, Protestant $-$ Catholic, is 0.1079}

(f) Is there a significant effect of religion on GPA (controlling for hours studied, age, and gender)? Cite a p-value in your answer.

\answer{2cm}{Yes -- p-value = 0.0260 -- partial F-test of $H_0: \beta_{religion1} = \beta_{religion2}=0$ (from \texttt{test} output)}

(g) Is there a significant difference in mean GPA between Protestant students and students of ``Other" religion? Cite a p-value in your answer.

\answer{2cm}{No -- p-value = 0.422 -- test of $H_0: \beta_{religion1}=0$}

(h) Is there a significant difference in mean GPA between Protestant students and Catholic students? Cite a p-value in your answer.

\answer{2cm}{No -- p-value = 0.167 -- test of $H_0: \beta_{religion1} - \beta_{religion2} = 0$ (from \texttt{lincom} output)}

(i) Explain how the degrees of freedom for the partial F-test that was conducted are calculated.

\answer{2cm}{Partial F-test: test statistic $F \sim F$(\# of $\beta$s being tested, DF for Residual for Full Model)\\
\# of $\beta$s being tested = 2 (\texttt{religion1})\\
DF for Residual for Full Model = 206 (in ANOVA table, on the ``Residual" line, under ``df" column)\\
$\rightarrow F$(2, 206)}

%%%
\newpage

\begin{stata}
. generate religion1 = (religion==1) if !missing(religion)
. generate religion2 = (religion==2) if !missing(religion)
. generate religion3 = (religion==3) if !missing(religion)

. regress gpa religion1 religion2 study age gender

      Source |       SS           df       MS      Number of obs   =       212
-------------+----------------------------------   F(5, 206)       =      8.91
       Model |  7.71521219         5  1.54304244   Prob > F        =    0.0000
    Residual |  35.6880173       206  .173242802   R-squared       =    0.1778
-------------+----------------------------------   Adj R-squared   =    0.1578
       Total |  43.4032295       211  .205702509   Root MSE        =    .41622

------------------------------------------------------------------------------
         gpa |      Coef.   Std. Err.      t    P>|t|     [95% Conf. Interval]
-------------+----------------------------------------------------------------
   religion1 |  -.0659408   .0818991    -0.81   0.422    -.2274086    .0955271
   religion2 |  -.1738371   .0647677    -2.68   0.008    -.3015297   -.0461445
       study |   .0123059   .0033299     3.70   0.000     .0057409     .018871
         age |   .0375648   .0095926     3.92   0.000     .0186526    .0564771
      gender |  -.1366602   .0584907    -2.34   0.020    -.2519773   -.0213431
       _cons |    2.00595   .2163566     9.27   0.000     1.579393    2.432507
------------------------------------------------------------------------------

. test religion1 religion2

 ( 1)  religion1 = 0
 ( 2)  religion2 = 0

       F(  2,   206) =    3.71
            Prob > F =    0.0260

. lincom religion1 - religion2

 ( 1)  religion1 - religion2 = 0

------------------------------------------------------------------------------
         gpa |      Coef.   Std. Err.      t    P>|t|     [95% Conf. Interval]
-------------+----------------------------------------------------------------
         (1) |   .1078964   .0778378     1.39   0.167    -.0455645    .2613572
------------------------------------------------------------------------------
\end{stata}

%%%
\newpage

\exercise{4}{
In the University of New Hampshire student survey, information was collected on the amount of alcohol students consumed. The result was a 33-point drinking scale score, where a higher score means more alcohol consumption. We are interested in whether there are differences in alcohol consumption among the years in school (\texttt{year}: 1=Freshman, 2=Sophomore, 3=Junior, 4=Senior), and also differences by sex  (\texttt{gender}; 1=male, 0=female).

A regression model was fit to predict drinking score (\texttt{drink}) using year in school and sex. Use the Stata output provided at the end of the problem (note: output spans 2 pages) to answer the questions below.
}

(a) Is there a significant effect of year in school on drinking score, adjusting for sex? Cite a p-value in your answer.

\answer{1.5cm}{Yes -- p-value = 0.0033 -- partial F-test of $H_0: \beta_{year2} = \beta_{year3}=\beta_{year4}=0$ (from \texttt{test} output)}

(b) Is there a significant effect of sex on drinking score, controlling for year in school?

\answer{1.5cm}{Yes -- p-value $<$ 0.0005 -- test of $H_0: \beta_{gender}=0$}

(c) Interpret the estimated coefficient for \texttt{year3} in this model

\answer{2cm}{The estimated mean drinking score for juniors is 0.176 points higher than for freshman, adjusting for sex.}

(d) Interpret the estimated coefficient for \texttt{gender} in this model

\answer{2cm}{The estimated mean drinking score for males is 3.78 points higher than for females, adjusting for year in school.}

%%%
\newpage

(e) Complete the table below with the estimated differences in means and the p-values for the comparisons. \emph{(Note: You may notice that this table contains a LOT of comparisons/p-values, with no adjustment for the fact that you are doing a lot of tests. Later in the course we will learn about the problem of ``multiple comparisons" and ways to adjust for the fact that doing lots of tests increases the chance you find something ``spurrious".)}

\begin{tabular}{ll | c | c | c}
& & Estimated Mean Difference, &  & \\
Group 1 & Group 2 & Group 2 $-$ Group 1 & 95\% CI for Difference & P-value\\ \hline
Freshman & Sophomore & & & \\
Freshman & Junior & & & \\
Freshman & Senior & & & \\
Sophomore & Junior & & & \\
Sophomore & Senior & & & \\
Junior & Senior & & & \\ \hline
\end{tabular}

\answer{1.5cm}{
\begin{tabular}{ll | c | c | c}
& & Estimated Mean Difference, &  & \\
Group 1 & Group 2 & Group 2 $-$ Group 1 & 95\% CI for Difference & P-value\\ \hline
Freshman & Sophomore & 1.85 & (-0.649, 4.34) & 0.146 \\
Freshman & Junior & 0.176 & (-2.25, 2.61) & 0.887 \\
Freshman & Senior & -2.31 & (-4.82, 0.193) & 0.070 \\
Sophomore & Junior & -1.67 & (-3.77, 0.431) & 0.119 \\
Sophomore & Senior & -4.16 & (-6.36, -1.96) & $<$0.0005 \\
Junior & Senior & -2.49 & (-4.61, -0.366) & 0.022 \\ \hline
\end{tabular}
}

(f) Rank the years in school by estimated mean drinking score (adjusted for sex), from largest to smallest (i.e., from drinks the most to the least). (Hint: use the estimated differences from the table.)

\answer{1.5cm}{From largest to smallest mean: Sophomores, Juniors, Freshman, Seniors}

%%%
\newpage

\begin{stata}
. generate year1 = (year==1) if !missing(year)
. generate year2 = (year==2) if !missing(year)
. generate year3 = (year==3) if !missing(year)
. generate year4 = (year==4) if !missing(year)

. regress drink year2 year3 year4 gender

      Source |       SS           df       MS      Number of obs   =       243
-------------+----------------------------------   F(4, 238)       =      9.59
       Model |  1517.94605         4  379.486512   Prob > F        =    0.0000
    Residual |  9417.27206       238    39.56837   R-squared       =    0.1388
-------------+----------------------------------   Adj R-squared   =    0.1243
       Total |  10935.2181       242  45.1868517   Root MSE        =    6.2903

------------------------------------------------------------------------------
       drink |      Coef.   Std. Err.      t    P>|t|     [95% Conf. Interval]
-------------+----------------------------------------------------------------
       year2 |   1.845409   1.266336     1.46   0.146    -.6492491    4.340067
       year3 |   .1760209   1.233302     0.14   0.887    -2.253561    2.605603
       year4 |   -2.31221   1.271725    -1.82   0.070    -4.817485    .1930656
      gender |   3.778905   .8144889     4.64   0.000     2.174377    5.383433
       _cons |   17.46344   1.046591    16.69   0.000     15.40167     19.5252
------------------------------------------------------------------------------

. test year2 year3 year4

 ( 1)  year2 = 0
 ( 2)  year3 = 0
 ( 3)  year4 = 0

       F(  3,   238) =    4.69
            Prob > F =    0.0033

. lincom year3 - year2

 ( 1)  - year2 + year3 = 0

------------------------------------------------------------------------------
       drink |      Coef.   Std. Err.      t    P>|t|     [95% Conf. Interval]
-------------+----------------------------------------------------------------
         (1) |  -1.669388   1.066031    -1.57   0.119     -3.76945    .4306743
------------------------------------------------------------------------------

. lincom year4 - year2

 ( 1)  - year2 + year4 = 0

------------------------------------------------------------------------------
       drink |      Coef.   Std. Err.      t    P>|t|     [95% Conf. Interval]
-------------+----------------------------------------------------------------
         (1) |  -4.157619   1.114837    -3.73   0.000    -6.353827    -1.96141
------------------------------------------------------------------------------

. lincom year4 - year3

 ( 1)  - year3 + year4 = 0

------------------------------------------------------------------------------
       drink |      Coef.   Std. Err.      t    P>|t|     [95% Conf. Interval]
-------------+----------------------------------------------------------------
         (1) |  -2.488231   1.077143    -2.31   0.022    -4.610182    -.366279
------------------------------------------------------------------------------
\end{stata}

%%%
\newpage

\exercise{4}{Information on 74 automobiles was collected (in 1978) to study the relationship between gas mileage (mpg) and various features of the cars. The predictors we would like to investigate are the weight of the car (\texttt{pounds}; lbs), price of the car (\texttt{price}; \$), and whether the car is made in the U.S. (\texttt{foreign}: 0=made in U.S.; 1=made outside U.S.).

We would like to know whether adding price and foreign-made status improve on a model that includes weight as the only predictor. Use the Stata output on the next 2 pages to answer this question. (Perform all steps of the appropriate hypothesis test.) Note that more output is provided than you need for the problem.
}

\answer{1.5cm}{
Full model: $E(MPG) = \beta_0 + \beta_1 WEIGHT + \beta_2 PRICE + \beta_3 FOREIGN$\\
Partial F-test needed\\
$H_0: \beta_2 = \beta_3 = 0$\\
$H_a: \beta_2 \ne 0$ and/or $\beta_3 \ne 0$\\
test statistic:\\
\begin{equation*}
F = \frac{\frac{MSS_{Full} - MSS_{Reduced}}{\text{\# of } \beta\text{s being tested}}}{\text{Mean Square Residual for Full Model}} = \frac{\frac{1620.30716 - 1591.9902}{2}}{11.7593185} = \frac{\frac{28.31696}{2}}{11.7593185} = \frac{14.15848}{11.7593185} = 1.20
\end{equation*}
Under $H_0$, $F \sim F($\# of $\beta$s being tested, $df_{RSS,Full}) = F(2, 70)$\\
p-value = $P(F_{(2,70)} > 1.20) = 0.31$ (Stata code: \texttt{display Ftail(2,70,1.20)})\\
Fail to reject $H_0$\\
There is no evidence that price and foreign-made status are significant predictors in a model that contains weight (p=0.31).}

%%%
\newpage

\begin{stata}
. regress mpg pounds

      Source |       SS           df       MS      Number of obs   =        74
-------------+----------------------------------   F(1, 72)        =    134.62
       Model |   1591.9902         1   1591.9902   Prob > F        =    0.0000
    Residual |  851.469256        72  11.8259619   R-squared       =    0.6515
-------------+----------------------------------   Adj R-squared   =    0.6467
       Total |  2443.45946        73  33.4720474   Root MSE        =    3.4389

------------------------------------------------------------------------------
         mpg |      Coef.   Std. Err.      t    P>|t|     [95% Conf. Interval]
-------------+----------------------------------------------------------------
      pounds |  -.0060087   .0005179   -11.60   0.000    -.0070411   -.0049763
       _cons |   39.44028   1.614003    24.44   0.000     36.22283    42.65774
------------------------------------------------------------------------------

. regress mpg pounds price

      Source |       SS           df       MS      Number of obs   =        74
-------------+----------------------------------   F(2, 71)        =     66.85
       Model |  1595.93249         2  797.966246   Prob > F        =    0.0000
    Residual |  847.526967        71  11.9369995   R-squared       =    0.6531
-------------+----------------------------------   Adj R-squared   =    0.6434
       Total |  2443.45946        73  33.4720474   Root MSE        =     3.455

------------------------------------------------------------------------------
         mpg |      Coef.   Std. Err.      t    P>|t|     [95% Conf. Interval]
-------------+----------------------------------------------------------------
      pounds |  -.0058175   .0006175    -9.42   0.000    -.0070489   -.0045862
       price |  -.0000935   .0001627    -0.57   0.567     -.000418    .0002309
       _cons |   39.43966   1.621563    24.32   0.000     36.20635    42.67296
------------------------------------------------------------------------------

. regress mpg pounds price foreign

      Source |       SS           df       MS      Number of obs   =        74
-------------+----------------------------------   F(3, 70)        =     45.93
       Model |  1620.30716         3  540.102388   Prob > F        =    0.0000
    Residual |  823.152295        70  11.7593185   R-squared       =    0.6631
-------------+----------------------------------   Adj R-squared   =    0.6487
       Total |  2443.45946        73  33.4720474   Root MSE        =    3.4292

------------------------------------------------------------------------------
         mpg |      Coef.   Std. Err.      t    P>|t|     [95% Conf. Interval]
-------------+----------------------------------------------------------------
      pounds |  -.0067758   .0009048    -7.49   0.000    -.0085805   -.0049712
       price |   .0000566   .0001922     0.29   0.769    -.0003268      .00044
     foreign |  -1.855891   1.289063    -1.44   0.154    -4.426846    .7150641
       _cons |   41.95948   2.377726    17.65   0.000     37.21725     46.7017
------------------------------------------------------------------------------
\end{stata}

\end{document}

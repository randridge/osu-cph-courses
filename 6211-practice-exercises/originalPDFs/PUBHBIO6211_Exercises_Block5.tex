%%% PUBHBIO 6211 - PRACTICE EXERCISES
\documentclass[17pt,landscape]{foils}

% Include command file
\usepackage{commandsExercises}
 
 %print numbers
%\toggletrue{solutions}
\togglefalse{solutions}

% Exercises
\begin{document}

\exercise{5}{
A survey of a random sample of students at the University of New Hampshire was conducted. We are interested in predictors of grade point average (GPA), which is measured on a 4-point scale.

We are interested in the relationship between GPA and alcohol consumption. Alcohol consumption is captured in the variable \texttt{drink}, where a higher number means more drinking. (Note: this drinking scale goes from 0=no drinking, to 33=max drinking score). We are curious as to whether the relationship between GPA and drinking is different for men and women. Thus, a regression model was fit using drinking score, sex (\texttt{gender}; 1=male, 0=female), and their interaction to predict GPA. Use the Stata output provided at the end of the problem to answer the questions below.
}

(a) What is the estimated intercept? Write a one-sentence interpretation of this quantity.

\answer{2.5cm}{$\hat{\beta}_0=3.24$\\
The estimated mean GPA for females with a drinking score of 0 is 3.24.}

(b) What is the estimated coefficient for the main effect of \texttt{drink}? Write a one-sentence interpretation of this quantity.

\answer{2.5cm}{$\hat{\beta}_{drink}=-0.0209$\\
For females, estimated mean GPA decreases by 0.0209 points for each 1 unit increase in drinking score.}

(c) What is the estimated coefficient for the main effect of \texttt{gender}? Write a one-sentence interpretation of this quantity.

\answer{2.5cm}{$\hat{\beta}_{gender}=-0.243$\\
Among students with a drinking score of 0, the estimated mean GPA for males is 0.243 points lower than the estimated mean for females.}

%%%
\newpage

(d) What is the estimated slope of drinking score for females? For males?

\answer{5cm}{Slope for females = $\hat{\beta}_{drink}=-0.0209$\\
Slope for males = $\hat{\beta}_{drink} + \hat{\beta}_{drink\_gender}=-0.0209+.00782=-0.013$}

(e) Among students who have a drinking score of 19 (the sample mean drinking score), what is the estimated difference in GPA between men and women? Indicate which group has a higher estimated GPA (men or women).

\answer{5cm}{Estimated difference, men $-$ women = $\hat{\beta}_{gender} + \hat{\beta}_{drink\_gender} \times DRINK = -.243 + .00782 \times 19 = -0.094$\\
Women have the higher estimated GPA.\\
~\\
Could also calculate each mean and subtract:\\
Women with drinking score = 19:\\
\qquad Estimated mean = $3.24 - 0.0209 \times 19 = 2.84$\\
Men with drinking score = 19:\\
\qquad Estimated mean = $3.24 - 0.0209 \times 19 - 0.243 \times 1 + 0.00782 \times 19 \times 1 = 2.75$\\
Difference, men $-$ women = $2.75 - 2.84 = -0.09$}

(f) Is there evidence that the relationship between GPA and drinking score is different for males and females? Cite specific evidence from the output. (Assume $\alpha=0.05$.)

\answer{2cm}{No, the interaction term (\texttt{drink\_gender}) is not significant, with p-value = 0.403}

%%%
\newpage

\begin{stata}
. generate drink_gender = drink*gender

. regress gpa drink gender drink_gender

      Source |       SS           df       MS      Number of obs   =       218
-------------+----------------------------------   F(3, 214)       =      7.23
       Model |  4.20817373         3  1.40272458   Prob > F        =    0.0001
    Residual |  41.5435654       214   .19412881   R-squared       =    0.0920
-------------+----------------------------------   Adj R-squared   =    0.0792
       Total |  45.7517391       217  .210837507   Root MSE        =     .4406

------------------------------------------------------------------------------
         gpa |      Coef.   Std. Err.      t    P>|t|     [95% Conf. Interval]
-------------+----------------------------------------------------------------
       drink |   -.020912   .0065085    -3.21   0.002     -.033741    -.008083
      gender |   -.243004   .1884557    -1.29   0.199    -.6144711    .1284631
drink_gender |   .0078177   .0093203     0.84   0.403    -.0105538    .0261891
       _cons |   3.237566   .1175289    27.55   0.000     3.005904    3.469229
------------------------------------------------------------------------------
\end{stata}

%%%
\newpage

\exercise{5}{
Data on standardized tests were recorded for 200 students. We are interested in predictors of writing score (\texttt{write}; higher is better; range is 31 to 67 in the dataset). In particular, we want to know whether sex (\texttt{female}; 1=female, 0=male) and social studies score (\texttt{socst}; higher is better; range is 26 to 71 in the dataset) are associated with writing score. We also think that the effect of social studies score on writing score might depend on sex. Thus, a regression model was fit using sex, social studies score, and their interaction to predict writing score. Use the Stata output provided at the end of the problem to answer the questions below.
}

(a) For females, what is the estimated regression line for social studies score predicting writing score?

\answer{3cm}{For females, $FEMALE = 1$\\
Estimated writing score $= 17.8 + 15.0 \times 1 + 0.625 \times SOCST - 0.205 \times 1 \times SOCST$\\
\qquad\qquad\qquad\qquad\qquad $= 32.8 + 0.42 \times SOCST$}

(a) For males, what is the estimated regression line for social studies score predicting writing score?

\answer{3cm}{For males, $FEMALE = 0$\\
Estimated writing score $= 17.8 + 15.0 \times 0 + 0.625 \times SOCST - 0.205 \times 0 \times SOCST$\\
\qquad\qquad\qquad\qquad\qquad $= 17.8 + 0.625 \times SOCST$}

(c) Is there a significant difference in the effect of social studies score on writing score for females versus males? Cite specific evidence from the output. (Assume $\alpha=0.05$.)

\answer{2.5cm}{Yes, the interaction term (\texttt{female\_socst}) is significant, with p-value = 0.033}

%%%
\newpage

(d) Is there a significant effect of social studies score on writing score for males? Cite specific evidence from the output. (Assume $\alpha=0.05$.)

\answer{2.5cm}{Yes, the main effect of social studies score ($\hat{\beta}_{socst}$) is significant, with p-value $<$ 0.0005 (Stata prints as 0.000)}

(e) Is there a significant effect of social studies score on writing score for females? Cite specific evidence from the output. (Assume $\alpha=0.05$.)

\answer{2.5cm}{Yes, the test for the linear combination of $\hat{\beta}_{socst} + \hat{\beta}_{female\_socst}$ is significant, with p-value $<$ 0.0005 (Stata prints as 0.000)}

(f) What is the estimated difference in mean writing score for females versus males, among students with a social studies score of 52 (approximately the sample mean)? Indicate which group has a higher estimated GPA (females or males).

\answer{4cm}{Estimated difference, females $-$ males = $\hat{\beta}_{female} + \hat{\beta}_{female\_socst} \times SOCST = 15.0 - 0.205 \times 52 = 4.34$\\
Females have the higher estimated writing score.}

(g) How do you interpret the intercept estimate in this model?

\answer{1.5cm}{The estimated mean writing score for males with a social studies score of 0 is 17.8.}

%%%
\newpage

\begin{stata}
. generate female_socst = female*socst

. regress write female socst female_socst

      Source |       SS           df       MS      Number of obs   =       200
-------------+----------------------------------   F(3, 196)       =     49.26
       Model |  7685.43528         3  2561.81176   Prob > F        =    0.0000
    Residual |  10193.4397       196  52.0073455   R-squared       =    0.4299
-------------+----------------------------------   Adj R-squared   =    0.4211
       Total |   17878.875       199   89.843593   Root MSE        =    7.2116

------------------------------------------------------------------------------
       write |      Coef.   Std. Err.      t    P>|t|     [95% Conf. Interval]
-------------+----------------------------------------------------------------
      female |   15.00001    5.09795     2.94   0.004     4.946132    25.05389
       socst |   .6247968   .0670709     9.32   0.000     .4925236    .7570701
female_socst |  -.2047288   .0953726    -2.15   0.033    -.3928171   -.0166405
       _cons |    17.7619   3.554993     5.00   0.000     10.75095    24.77284
------------------------------------------------------------------------------

. lincom socst + female_socst

 ( 1)  socst + female_socst = 0

------------------------------------------------------------------------------
       write |      Coef.   Std. Err.      t    P>|t|     [95% Conf. Interval]
-------------+----------------------------------------------------------------
         (1) |    .420068   .0678044     6.20   0.000     .2863482    .5537878
------------------------------------------------------------------------------
\end{stata}

%%%
\newpage

\exercise{5}{
A survey of a random sample of students at the University of New Hampshire was conducted. We are interested in predictors of grade point average (GPA), which is measured on a 4-point scale.

For this question, we are interested in whether a student has a job or not impacts GPA (\texttt{job}: 1=has job, 0=does not). We are also interested in whether being in a fraternity/sorority (\texttt{frat}: 1=in fraternity/sorority, 0=not) might modify this effect. Thus, a regression model was fit using having a job, being in a fraternity/sorority, and their interaction was used to predict GPA. Use the Stata output provided at the end of the problem to answer the questions below.
}

(a) What is the estimated intercept? Write a one-sentence interpretation of this quantity.

\answer{2.5cm}{$\hat{\beta}_0=2.79$\\
The estimated mean GPA for students without a job and who are not in a fraternity/sorority is 2.79.}

(b) What is the estimated coefficient for the main effect of \texttt{job}? Write a one-sentence interpretation of this quantity.

\answer{2.5cm}{$\hat{\beta}_{job}=0.05799$\\
For students who are not in a fraternity/sorority, the estimated mean GPA for students with a job is 0.05799 points \underline{higher} than for students without a job.}

(c) What is the estimated coefficient for the main effect of \texttt{frat}? Write a one-sentence interpretation of this quantity.

\answer{2.5cm}{$\hat{\beta}_{frat}=-0.220$\\
For students who do not have a job, the estimated mean GPA for students in a fraternity/sorority is 0.220 points \underline{lower} than for students not in a fraternity/sorority.}

%%%
\newpage

(d) Calculate the estimated mean GPA for each group of students (e.g., ``no job, not in fraternity/sorority", ``has job, not in fraternity/sorority", etc.).

\answer{4cm}{
no job, not in fraternity/sorority: Estimated mean = $2.79$\\
has job, not in fraternity/sorority: Estimated mean = $2.79 + 0.05799 = 2.85$\\
no job, in fraternity/sorority: Estimated mean = $2.79 - 0.220 = 2.57$\\
has job, in fraternity/sorority: Estimated mean = $2.79 + 0.05799 - 0.220 + 0.275 = 2.90$
}

(e) What is the estimated difference in GPA for students who work compared to students who do not work, for students not in a fraternity/sorority?

\answer{2.5cm}{Difference is $\hat{\beta}_{job}=0.05799$\\
Could also subtract means as calculated in (d): $2.85 - 2.79 = 0.06$}

(f) What is the estimated difference in GPA for students who work compared to students who do not work, for students who are in a fraternity/sorority?

\answer{2.5cm}{Difference is $\hat{\beta}_{job} + \hat{\beta}_{job\_frat}=0.05799+0.2746 = 0.333$\\
Could also subtract means as calculated in (d): $2.90 - 2.57 = 0.33$}

(g) Is there evidence that the effect of having a job on GPA is significantly different for students who are in a fraternity/sorority compared to students not in a fraternity/sorority? Cite specific evidence from the output. (Assume $\alpha=0.05$.)

\answer{1.5cm}{No, the interaction term (\texttt{job\_frat}) is not significant, with p-value = 0.075 (but, close!)}

%%%
\newpage

\begin{stata}
. generate job_frat = job*frat

. regress gpa job frat job_frat

      Source |       SS           df       MS      Number of obs   =       216
-------------+----------------------------------   F(3, 212)       =      2.89
       Model |  1.78161624         3   .59387208   Prob > F        =    0.0363
    Residual |  43.5312053       212  .205335874   R-squared       =    0.0393
-------------+----------------------------------   Adj R-squared   =    0.0257
       Total |  45.3128216       215   .21075731   Root MSE        =    .45314

------------------------------------------------------------------------------
         gpa |      Coef.   Std. Err.      t    P>|t|     [95% Conf. Interval]
-------------+----------------------------------------------------------------
         job |   .0579913   .0713873     0.81   0.418    -.0827285    .1987111
        frat |  -.2200616   .1066417    -2.06   0.040    -.4302755   -.0098477
    job_frat |   .2746088    .153546     1.79   0.075    -.0280637    .5772812
       _cons |   2.794462   .0562051    49.72   0.000     2.683669    2.905254
------------------------------------------------------------------------------
\end{stata}

%%%
\newpage

\exercise{5}{
Data on standardized tests were recorded for 200 students. Researchers are interested in the relationship between reading score (\texttt{read}) and both math score (\texttt{math}) and social studies score (\texttt{socst}). For these scores, higher is better (summary statistics shown in Stata output). A regression model was fit using math score, social studies score, and their interaction to predict reading score. Use the Stata output provided at the end of the problem to answer the questions below.
}

(a) What is the estimated intercept? Write a one-sentence interpretation of this quantity.

\answer{2.5cm}{$\hat{\beta}_0=37.8$\\
The estimated mean reading score for students a math score of 0 and a social studies score of 0 is 37.8.}

(b) What is the estimated coefficient for the main effect of \texttt{math}? Write a one-sentence interpretation of this quantity.

\answer{2.5cm}{$\hat{\beta}_{math}=-0.111$\\
For students with a social studies score of 0, the estimated mean reading score \underline{decreases} by 0.111 points for each 1 point increase in math score.}

(c) What is the estimated coefficient for the main effect of \texttt{socst}? Write a one-sentence interpretation of this quantity.

\answer{2.5cm}{$\hat{\beta}_{socst}=-0.220$\\
For students with a math score of 0, the estimated mean reading score \underline{decreases} by 0.220 points for each 1 point increase in social studies score.}

%%%
\newpage

(d) Does the association between math score and reading score significantly depend on social studies score? Cite specific evidence from the output. (Assume $\alpha=0.05$.)

\answer{2cm}{Yes, the interaction term (\texttt{math\_socst}) is significant, with p-value = 0.032}

(e) What is the estimated slope of math score for students whose social studies score is 52 (approximately the sample mean)?

\answer{2cm}{slope for math = $\hat{\beta}_{math} + \hat{\beta}_{math\_socst} \times SOCST = -0.111 + 0.0113 \times 52 = 0.48$}

(f) What is the estimated slope of math score for students whose social studies score is 62 (approximately 1 SD above the sample mean)?

\answer{2cm}{slope for math = $\hat{\beta}_{math} + \hat{\beta}_{math\_socst} \times SOCST = -0.111 + 0.0113 \times 62 = 0.59$}

(g) Explain how the effect of math score on reading score is impacted by social studies score. (Using your answers to parts (e) and (f) may be helpful.)

\answer{2cm}{The slope of math score on reading score gets larger (more positive) as social studies score increases. So the effect of math score on reading score is \underline{stronger} for students with higher social studies scores.}

(h) What about the effect of social studies on reading score? How is it affected by math score?

\answer{2cm}{The slope of social studies score on reading score gets larger (more positive) as math score increases. So the effect of social studies score on reading score is \underline{stronger} for students with higher math scores.}

%%%
\newpage

\begin{stata}
. summarize read math socst

    Variable |        Obs        Mean    Std. Dev.       Min        Max
-------------+---------------------------------------------------------
        read |        200       52.23    10.25294         28         76
        math |        200      52.645    9.368448         33         75
       socst |        200      52.405    10.73579         26         71

. generate math_socst = math*socst

. regress read math socst math_socst

      Source |       SS           df       MS      Number of obs   =       200
-------------+----------------------------------   F(3, 196)       =     78.61
       Model |  11424.7622         3  3808.25406   Prob > F        =    0.0000
    Residual |  9494.65783       196  48.4421318   R-squared       =    0.5461
-------------+----------------------------------   Adj R-squared   =    0.5392
       Total |    20919.42       199  105.122714   Root MSE        =      6.96

------------------------------------------------------------------------------
        read |      Coef.   Std. Err.      t    P>|t|     [95% Conf. Interval]
-------------+----------------------------------------------------------------
        math |  -.1105123   .2916338    -0.38   0.705    -.6856552    .4646307
       socst |  -.2200442   .2717539    -0.81   0.419    -.7559812    .3158928
  math_socst |   .0112807   .0052294     2.16   0.032     .0009677    .0215938
       _cons |   37.84271   14.54521     2.60   0.010     9.157506    66.52792
------------------------------------------------------------------------------
\end{stata}

%%%
\newpage

\exercise{5}{
The model from the previous exercise was rerun, centering the predictors by subtracting their approximate sample means. Use the Stata output provided at the end of the problem to answer the questions below.
}

(a) What is the estimated intercept? Write a one-sentence interpretation of this quantity.

\answer{3.5cm}{$\hat{\beta}_0=51.2$\\
The estimated mean reading score for students a math score of 52 and a social studies score of 52 is 51.2.}

(b) What is the estimated coefficient for the main effect of \texttt{math\_52}? Write a one-sentence interpretation of this quantity.

\answer{3.5cm}{$\hat{\beta}_{math\_52}=0.476$\\
For students with a social studies score of 52, the estimated mean reading score \underline{increases} by 0.476 points for each 1 point increase in math score.}

(c) What is the estimated coefficient for the main effect of \texttt{socst\_52}? Write a one-sentence interpretation of this quantity.

\answer{3.5cm}{$\hat{\beta}_{socst\_52}=0.367$\\
For students with a math score of 52, the estimated mean reading score \underline{increases} by 0.367 points for each 1 point increase in social studies score.}

%%%
\newpage

\begin{stata}
. generate math_52 = math - 52
. generate socst_52 = socst - 52
. generate math_52_socst_52 = math_52*socst_52

. regress read math_52 socst_52 math_52_socst_52

      Source |       SS           df       MS      Number of obs   =       200
-------------+----------------------------------   F(3, 196)       =     78.61
       Model |  11424.7622         3  3808.25406   Prob > F        =    0.0000
    Residual |  9494.65783       196  48.4421318   R-squared       =    0.5461
-------------+----------------------------------   Adj R-squared   =    0.5392
       Total |    20919.42       199  105.122714   Root MSE        =      6.96

----------------------------------------------------------------------------------
            read |      Coef.   Std. Err.      t    P>|t|     [95% Conf. Interval]
-----------------+----------------------------------------------------------------
         math_52 |   .4760853   .0640923     7.43   0.000     .3496862    .6024843
        socst_52 |   .3665534    .055092     6.65   0.000     .2579042    .4752025
math_52_socst_52 |   .0112807   .0052294     2.16   0.032     .0009677    .0215938
           _cons |   51.15685   .5674107    90.16   0.000     50.03784    52.27587
----------------------------------------------------------------------------------
\end{stata}

\end{document}

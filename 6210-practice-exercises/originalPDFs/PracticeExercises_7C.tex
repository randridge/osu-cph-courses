%%% PUBHBIO 6210 - PRACTICE EXERCISES
\documentclass[17pt,landscape]{foils}

% Include command file
\usepackage{commandsExercises}
 
 %print numbers
%\toggletrue{solutions}
\togglefalse{solutions}

% Exercises
\begin{document}

\exercise{7C}{Linear regression was used to investigate the relationship between age (years) and systolic blood pressure (mmHg) in a sample of adults ranging from 29 to 69 years old. Stata output is below.}

\begin{stata}
. regress sbp age

      Source |       SS       df       MS              Number of obs =      29
-------------+------------------------------           F(  1,    27) =   66.81
       Model |  6110.10173     1  6110.10173           Prob > F      =  0.0000
    Residual |  2469.34654    27  91.4572794           R-squared     =  0.7122
-------------+------------------------------           Adj R-squared =  0.7015
       Total |  8579.44828    28  306.408867           Root MSE      =  9.5633

------------------------------------------------------------------------------
         sbp |      Coef.   Std. Err.      t    P>|t|     [95% Conf. Interval]
-------------+----------------------------------------------------------------
         age |   .9493225   .1161445     8.17   0.000     .7110137    1.187631
       _cons |   97.07708   5.527552    17.56   0.000     85.73549    108.4187
------------------------------------------------------------------------------
\end{stata}

(a) Calculate the sample correlation between age and SBP.

\answer{1.5cm}{$r = \text{sign}(\hat{\beta}_1) \times \sqrt{R^2} = + \sqrt{0.7122} = 0.844$}

(b) If we convert age from years to months, will the sample correlation change? Will the slope estimate change? Will the $R^2$ change?

\answer{1.5cm}{Correlation will not change, slope will change, $R^2$ will not change.}

%%%
\newpage

\exercise{7C}{A summary of some of the variables in the New Hampshire student survey data is on the next page. Use this to answer the questions.}

(a) If I perform a linear regression using the miles away from school a student lives (\texttt{miles}) to predict his/her GPA (\texttt{gpa}), will the slope estimate be positive or negative?

\answer{1.5cm}{positive, because $r=0.1558 >0$}

(b) If I perform a linear regression using the miles away from school a student lives (\texttt{miles}) to predict his/her GPA (\texttt{gpa}), what will the coefficient of determination for the regression be?

\answer{1.5cm}{$R^2 = r^2 =0.1558^2 =  0.0243$}

(c) If I perform a linear regression using the miles away from school a student lives (\texttt{miles}) to predict his/her drinking score (\texttt{drink}), will the slope estimate be positive or negative?

\answer{1.5cm}{negative, because $r=-0.2702 <0$}

(d) If I perform a linear regression using the miles away from school a student lives (\texttt{miles}) to predict his/her drinking score (\texttt{drink}), what will the coefficient of determination for the regression be?

\answer{1.5cm}{$R^2 = r^2 =(-0.2702)^2 =  0.073$}

(e) If I perform a linear regression using drinking score (\texttt{drink} to predict the miles away from school a student lives (\texttt{miles}), what will the coefficient of determination for the regression be?

\answer{1.5cm}{Same as in (d), $R^2=0.073$ -- switching X and Y doesn't change $r$ or $R^2$}

%%%
\newpage

Output for problem on previous page:
\begin{stata}
. corr gpa drink miles
(obs=206)

             |      gpa    drink    miles
-------------+---------------------------
         gpa |   1.0000
       drink |  -0.2591   1.0000
       miles |   0.1558  -0.2702   1.0000
\end{stata}

\end{document}

%%% PUBHBIO 6210 - PRACTICE EXERCISES
\documentclass[17pt,landscape]{foils}

% Include command file
\usepackage{commandsExercises}
 
 %print numbers
%\toggletrue{solutions}
\togglefalse{solutions}

% Exercises
\begin{document}

\exercise{3B}{Answer the following questions:

(a) What kind of data is best summarized by a histogram?

\answer{1.5cm}{ANSWER: discrete and continuous}

(b) What kind of data can be summarized by bar charts?  

\answer{1.5cm}{ANSWER: nominal and ordinal}

(c) Are the data shown in the histogram below symmetric, right skewed, left skewed, or none of those?

\includegraphics[height=2in]{./figures/3A1}

\answer{1.5cm}{ANSWER: right skewed}

(d) For the histogram above, which would we expect to be larger -- the mean or the median?

\answer{1.5cm}{ANSWER: mean, since for right-skewed data the mean $>$ median}
}

%%%
\newpage

\exercise{3B}{A small sample of undergraduate students was randomly selected from a chemistry course. Their GPAs were recorded and are shown below. Use these data to calculate the requested summary measures.

GPAs: 3.08, 2.5, 3.33, 3.75, 3.1, 2.64

(a) Mean

\answer{3cm}{ANSWER: 3.07\\
~\\
$\displaystyle \bar{x} = \frac{3.08+ 2.5+ 3.33+ 3.75+ 3.1+ 2.64}{6} = \frac{18.4}{6}=3.07$}

(b) Median

\answer{3cm}{ANSWER: 3.09\\
~\\
Sort data values: 2.5, 2.64, 3.08, 3.1, 3.33, 3.75\\
Even number of values (6) so the median is the average of the middle two: $\displaystyle \frac{3.08+3.1}{2}=3.09$}

(c) Geometric mean

\answer{1.5cm}{ANSWER: 3.04\\
~\\
Log-transform (natural log being used here):\\[5pt]
 \{ln(3.08), ln(2.5), ln(3.33), ln(3.75), ln(3.1), ln(2.64)\} = \{1.125, 0.916, 1.203, 1.322, 1.131, 0.971\}\\[5pt]
Take (arithmetic) mean of these values: $\displaystyle \frac{1.125+0.916+\dots+0.971}{6} = \frac{6.668}{6}=1.111$\\[5pt]
Back-transform: $ e^{1.111}= 3.04$
}
}

\end{document}

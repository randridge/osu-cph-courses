%%% PUBHBIO 6210 - PRACTICE EXERCISES
\documentclass[17pt,landscape]{foils}

% Include command file
\usepackage{commandsExercises}

 %print numbers
%\toggletrue{solutions}
\togglefalse{solutions}

% Exercises
\begin{document}

\exercise{3A}{In the New Hampshire student survey data, one of the variables collected was where the student lives. Complete the table below with the appropriate summaries (as labeled at the top of the columns).

\answer{1cm}{ANSWER: (see below)\\
~\\
\begin{tabular}{rrrr}
Guessed & & & \\
grades & Freq. & Percent & Cum.\\ \hline
    As \& Bs &        110   &    110/242=0.45   &    110/242=0.45\\
    Bs \& Cs &        120   &    120/242=0.50   &    (110+120)/242=0.95\\
    Cs \& Ds &         12    &    12/242=0.050   &   (110+120+12)/242=1\\ \hline
    Total & 242 & 100 & -- \\
\end{tabular}
}}

\begin{stata}
. tab grades

    Guessed |
grades this |
   semester |      Freq.     Percent        Cum.
------------+-----------------------------------
    As & Bs |        110
    Bs & Cs |        120
    Cs & Ds |         12
------------+-----------------------------------
      Total |        242      100.00         --
\end{stata}

%%%
\newpage

\exercise{3A}{In a study of asthma in minority youth in Detroit, Michigan, two pieces of information collected about the children were their asthma severity (3 levels: intermittent, mild persistent, moderate/severe persistent) and their family income. These data were obtained and summarized in the table below:

\begin{tabular}{lccc}
 & \multicolumn{3}{c}{Asthma Severity}\\ \cmidrule{2-4}
Income & Intermittent & Mild Persistent & Moderate/Severe Persistent \\ \hline
$<$\$15,000 		& 71 (43\%) 	& 41 (53\%)	& 34 (49\%)	\\
\$15,000-\$40,000 	& 66 (40\%) 	& 27 (35\%)	& 29 (41\%)	\\
\$40,000+ 		& 28 (17\%)	& 10 (13\%)	& 7 (10\%)	\\ \hline
\end{tabular}

(a) Are the percentages in this table overall percentages, row percentages, or column percentages?

\answer{2cm}{ANSWER: Column percentages\\
~\\
They sum to 100\% down the row, e.g., 43\% + 40\% + 17\% = 100\% (Note: Column 2 percentages sum to 101\% due to rounding)}

(b) Why might the investigators have chosen to use these percentages (instead of the other options)?

\answer{3cm}{ANSWER: The column percentages allow us to easily compare the distribution of income across the three asthma severity groups. So we could see whether kids with more severe asthma tended to have lower incomes (for example).}

(c) If one child with intermittent asthma is randomly selected from this study, what is the probability that he has the highest level of household income (\$40,000+)?

\answer{1cm}{ANSWER: 0.17\\
~\\
Read the column percent right off the table, or calculate as $28/(71+66+28)=28/156=0.17$}
}

\end{document}
